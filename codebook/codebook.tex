%!TEX TS-program = xelatex
%!TEX encoding = UTF-8 Unicode
\documentclass[a4paper,10pt,twocolumn,oneside]{article}
\setlength{\columnsep}{10pt}                                                                    %兩欄模式的間距
\setlength{\columnseprule}{0pt}                                                                %兩欄模式間格線粗細

\usepackage{amsthm}								%定義,例題
\usepackage{amssymb}
%\usepackage[margin=2cm]{geometry}
%\usepackage{fontspec}								%設定字體
\usepackage{color}
\usepackage[x11names]{xcolor}
\usepackage{xeCJK}								%xeCJK
\usepackage{listings}								%顯示code用的
%\usepackage[Glenn]{fncychap}						%排版,頁面模板
\usepackage{fancyhdr}								%設定頁首頁尾
\usepackage{graphicx}								%Graphic
\usepackage{enumerate}
\usepackage{titlesec}
\usepackage{amsmath}
\usepackage{pdfpages}
%\usepackage[T1]{fontenc}
\usepackage{amsmath, courier, listings, fancyhdr, graphicx}
\topmargin=0pt
\headsep=5pt
\textheight=780pt
\footskip=0pt
\voffset=-40pt
\textwidth=545pt
\marginparsep=0pt
\marginparwidth=0pt
\marginparpush=0pt
\oddsidemargin=0pt
\evensidemargin=0pt
\hoffset=-42pt

%\renewcommand\listfigurename{圖目錄}
%\renewcommand\listtablename{表目錄} 

%%%%%%%%%%%%%%%%%%%%%%%%%%%%%

\setmainfont{Monaco}				%主要字型
\setCJKmainfont{儷宋 Pro}			%中文字型
%\setmainfont{Linux Libertine G}
\setmonofont{Monaco}
%\setmainfont{sourcecodepro}
\XeTeXlinebreaklocale "zh"						%中文自動換行
\XeTeXlinebreakskip = 0pt plus 1pt				%設定段落之間的距離
\setcounter{secnumdepth}{3}						%目錄顯示第三層

%%%%%%%%%%%%%%%%%%%%%%%%%%%%%
\makeatletter
\lst@CCPutMacro\lst@ProcessOther {"2D}{\lst@ttfamily{-{}}{-{}}}
\@empty\z@\@empty
\makeatother
\lstset{											% Code顯示
language=C++,										% the language of the code
basicstyle=\footnotesize\ttfamily, 						% the size of the fonts that are used for the code
%numbers=left,										% where to put the line-numbers
numberstyle=\footnotesize,						% the size of the fonts that are used for the line-numbers
stepnumber=1,										% the step between two line-numbers. If it's 1, each line  will be numbered
numbersep=5pt,										% how far the line-numbers are from the code
backgroundcolor=\color{white},					% choose the background color. You must add \usepackage{color}
showspaces=false,									% show spaces adding particular underscores
showstringspaces=false,							% underline spaces within strings
showtabs=false,									% show tabs within strings adding particular underscores
frame=false,											% adds a frame around the code
tabsize=2,											% sets default tabsize to 2 spaces
captionpos=b,										% sets the caption-position to bottom
breaklines=true,									% sets automatic line breaking
breakatwhitespace=false,							% sets if automatic breaks should only happen at whitespace
escapeinside={\%*}{*)},							% if you want to add a comment within your code
morekeywords={*},									% if you want to add more keywords to the set
keywordstyle=\bfseries\color{Blue1},
commentstyle=\itshape\color{Red4},
stringstyle=\itshape\color{Green4},
}

%%%%%%%%%%%%%%%%%%%%%%%%%%%%%

\begin{document}
\pagestyle{fancy}
\fancyfoot{}
%\fancyfoot[R]{\includegraphics[width=20pt]{ironwood.jpg}}
\fancyhead[L]{National Chiao Tung University Ragnarok}
\fancyhead[R]{\thepage}
\renewcommand{\headrulewidth}{0.4pt}
\renewcommand{\contentsname}{Contents} 

\scriptsize
\tableofcontents
%%%%%%%%%%%%%%%%%%%%%%%%%%%%%

\newpage

\section{Basic}
\subsection{vimrc}
\lstinputlisting{Basic/vimrc}
\subsection{default}
\lstinputlisting{Basic/default.cpp}
\subsection{FastInput}
\lstinputlisting{Basic/fastinput.cpp}

\newpage
\section{Data Structure}
\subsection{Disjoint Set}
\lstinputlisting{Data-Structure/Disjoint-Set.cpp}
\subsection{Segment Tree}
\lstinputlisting{Data-Structure/Segment-Tree.cpp}
\subsection{Treap}
\lstinputlisting{Data-Structure/Treap.cpp}
\subsection{monotonic-queue}
\lstinputlisting{Data-Structure/mqueue.cpp}
\newpage

\section{Graph}
\subsection{BCC}
\lstinputlisting{Graph/BCC.cpp}
\subsection{SCC}
\lstinputlisting{Graph/SCC.cpp}
\subsection{SPFA}
\lstinputlisting{Graph/SPFA.cpp}
\subsection{Dijkstra}
\lstinputlisting{Graph/Dijkstra.cpp}
\subsection{Floyd-Warshall}
\lstinputlisting{Graph/floydwarshall.cpp}
\subsection{Bipartite Match}
\lstinputlisting{Graph/Bipartite-Match.cpp}
\subsection{Directed-MST}
\lstinputlisting{Graph/DMST.cpp}
\subsection{LCA}
\lstinputlisting{Graph/LCA.cpp}
\subsection{LCA2}
\lstinputlisting{Graph/LCA2.cpp}
\subsection{MST-Prim}
\lstinputlisting{Graph/Prim.cpp}
\subsection{MST-kruskal}
\lstinputlisting{Graph/mst-kruskal.cpp}
\subsection{Manhattan-Mst}
\lstinputlisting{Graph/manhattan-mst.cpp}
\subsection{Flow-Dicnic}
\lstinputlisting{Graph/flow-dicnic.cpp}
\subsection{Flow-MinCost}
\lstinputlisting{Graph/flow-mincost.cpp}
\subsection{HeavyLight-Decomposition}
\lstinputlisting{Graph/heavylight-decomposition.cpp}
\newpage

\section{String Theory}
\subsection{KMP}
\lstinputlisting{String-Theory/KMP.cpp}
\subsection{Z}
\lstinputlisting{String-Theory/Z.cpp}
\subsection{Trie}
\lstinputlisting{String-Theory/Trie.cpp}
\subsection{AC automaton}
\lstinputlisting{String-Theory/AC-automaton.cpp}
\subsection{Suffix Array}
\lstinputlisting{String-Theory/Suffix-Array.cpp}
\newpage

\section{Geometry}
\subsection{Point}
\lstinputlisting{Geometry/Point.cpp}
\newpage

\section{Sort}
\subsection{Heap Sort}
\lstinputlisting{Sort/Heap-Sort.cpp}
\subsection{Merge Sort}
\lstinputlisting{Sort/Merge-Sort.cpp}
\subsection{Radix Sort}
\lstinputlisting{Sort/Radix-Sort.cpp}
\subsection{Shell Sort}
\lstinputlisting{Sort/Shell-Sort.cpp}
\newpage
\section{Math}
\subsection{Extended Euclidean}
\lstinputlisting{Math/Extended-Euclidean.cpp}
\subsection{Prime}
\lstinputlisting{Math/Prime.cpp}
\subsection{Factor Decomposition}
\lstinputlisting{Math/Factor-Decomposition.cpp}
\subsection{Module Inverse}
\lstinputlisting{Math/Module-Inverse.cpp}
\subsection{Phi}
\lstinputlisting{Math/Phi.cpp}
\subsection{Miller Rabin}
\lstinputlisting{Math/Miller-Rabin.cpp}
\subsection{FFT}
\lstinputlisting{Math/FFT.cpp}
\subsection{Fraction}
\lstinputlisting{Math/Fraction.cpp}
\subsection{Matrix}
\lstinputlisting{Math/Matrix.cpp}
\subsection{BigInt}
\lstinputlisting{Math/BigInt.cpp}
\newpage

\clearpage

\end{document}
